% DOCUMENT FORMATING
\documentclass[12pt]{article}
\usepackage[margin=1in]{geometry}

% PACKAGES
\usepackage{amsmath} % For extended formatting
\usepackage{amssymb} % For math symbols
\usepackage{amsthm} % For proof environment
\usepackage{array} % For tables
\usepackage{enumerate} % For lists
\usepackage{extramarks} % For headers and footers
\usepackage{fancyhdr} % For custom headers
\usepackage{graphicx} % For inserting images
\usepackage{multicol} % For multiple columns
\usepackage{verbatim} % For displaying code
\usepackage{tkz-euclide}
\usepackage{pgfplots}

% SET UP HEADER AND FOOTER
\pagestyle{fancy}
\lhead{\MyCourse} % Top left header
\chead{\MyTopicTitle} % Top center header
\rhead{\MyAssignment} % Top right header
\lfoot{\MyCampus} % Bottom left footer
\cfoot{} % Bottom center footer
\rfoot{\MySemester} % Bottom right footer
\renewcommand\headrulewidth{0.4pt} % Size of the header rule
\renewcommand\footrulewidth{0.4pt} % Size of the footer rule
% ----------
% TITLES AND NAMES 
% ----------

\newcommand{\MyCourse}{Math 321}
\newcommand{\MyTopicTitle}{Homework 1 part 2}
\newcommand{\MyAssignment}{Shinya Saito\qquad \qquad \qquad}
\newcommand{\MySemester}{Fall 2019}
\newcommand{\MyCampus}{University of Hawaii at Manoa}



\begin{document}
\textbf{1b} You won't go skiing, or you will and there won't be any snow. \newline
Let sk represent, "You will go skiing." \newline
Let sn represent, "There will be snow." \newline 
\textbf{Answer}: $\neg$sk $\vee$(sk$\wedge \neg$sn) \newline 
\vspace{0.5cm}

\textbf{2b}  I'll have either fish or chicken, but I won't have both fish and mashed potatoes. \newline 
Let c represent the phrase, "I will have chicken" \newline 
Let p represent the phrase, "I will have mashed potatoes". \newline
Let f represent the phrase "I will have fish". \newline
\textbf{Answer} (f $\vee$ c ) $\wedge \neg$(f $\wedge$ p) \newline

\textbf{3}
a) Alice and Bob are not both in the room. \newline
Let A represent the phrase,  "Alice is in the room." \newline
Let B represent the phrase, "Bob is in the room." \newline
\textbf{Answer}: $\neg$(A$\wedge$ B) \newline
b) Alice and Bob are both not in the room. \newline 
If Alice is in the room Bob is not and so vice versa.\newline 
Let A represent the phrase,  "Alice is in the room." \newline
Let B represent the phrase, "Bob is in the room." \newline
$\neg$ A $\wedge \neg$ B \newline
c) Either Alice or Bob are not in the room. \newline 
Let A represent the phrase,  "Alice is in the room." \newline
Let B represent the phrase, "Bob is in the room." \newline
\textbf{Answer}: ($\neg$ A) $\vee$ ($\neg$ B) \newline 
d)Neither Alice or Bob are not in the room. \newline
Let A represent the phrase,  "Alice is in the room." \newline
Let B represent the phrase, "Bob is in the room." \newline
\textbf{Answer}: $\neg$ (A $\vee$ B) \newline 
\vspace{0.5cm} 

\textbf{4} 
a) $\neg (\neg P \vee \neg \neg R)$ \newline 
$\neg \neg$ R has the same meaning of R by double negation \newline 
$\neg P$ by negation rule means not P \newline 
$\neg (\neg P \vee \neg \neg R) \equiv \neg(\neg P \vee R)$  \newline 

This formula is well formed as the meaning still upholds even when the different rules are applied. \newline
b) $\neg (P,Q, \wedge R)$ \newline 
Based on what we are given, a (,) symbols is not a valid connectives, so therefore, this is not a well formed formula. \newline 
c) P $\wedge \neg$ P\newline 
This is a well formed formula as it has a comparison of a negation and itself. \newline
d) (P $\vee$ Q)(P $\vee$R) \newline 
Since there is no valid connectives between two statements this is not a valid one as you cannot multiply statements together. \newline
\vspace{0.5cm} 
\textbf{6a} (S $\vee$ G) $\wedge$ ($\neg$ S $\vee \neg G)$\newline 
S $\rightarrow$ Steve is happy. \newline 
G $\rightarrow$ George is happy. \newline 
Since there is an and sign between two reasoning, that means one has to be happy and the other is not happy. \newline 
Therefore in the English language: Either Steve or George is happy, and either Steve or George is not happy.  

\end{document}
